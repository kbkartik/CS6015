\documentclass[solution,addpoints,12pt]{exam}
\usepackage{amsmath}
\usepackage{amsthm}
\usepackage{amssymb}
\usepackage{tikz}
\usepackage{animate}
\usepackage{hyperref}

\newtheorem{theorem}{Theorem}
\newtheorem{lemma}[theorem]{Lemma}

\newenvironment{Solution}{\begin{EnvFullwidth}\begin{solution}}{\end{solution}\end{EnvFullwidth}}

\printanswers
%\unframedsolutions
\pagestyle{headandfoot}

%%%%%%%%%%%%%%%%%%%%%%%%%%%%%%%%%%%%%%%%%%%%%%%%%%%%%%
%%%%%%%%%%%%%%%%%%% INSTRUCTIONS %%%%%%%%%%%%%%%%%%%%%
% * Fill in your name and roll number below

% * Answer in place (after each question)

% * Use \begin{solution} and \end{solution} to typeset
%   your answers.
%%%%%%%%%%%%%%%%%%%%%%%%%%%%%%%%%%%%%%%%%%%%%%%%%%%%%%
%%%%%%%%%%%%%%%%%%%%%%%%%%%%%%%%%%%%%%%%%%%%%%%%%%%%%%

% Fill in the details below
\def\studentName{\textbf{Name: TODO}}
\def\studentRoll{\textbf{Roll No: TODO}}

\firstpageheader{CS 6015 (LARP) - Homework 2}{}{\studentName,\studentRoll}
\firstpageheadrule

\newcommand{\brac}[1]{\left[ #1 \right]}
\newcommand{\curly}[1]{\left\{ #1 \right\}}
\newcommand{\paren}[1]{\left( #1 \right)}
\newcommand{\card}[1]{\left\lvert #1 \right\rvert}

\begin{document}

\noindent \textbf{Honor code}: I pledge on my honor that: I have completed all steps in the below homework on my own, I have not used any unauthorized materials while completing this homework, and I have not given anyone else access to my homework.
\\~\\~\\
\begin{flushright}
\textbf{Name and Signature}

\end{flushright}


\begin{questions}

\question[1] Have you read and understood the honor code?
\begin{solution}

\end{solution}

\uplevel{\textbf{Concept}:  Linear Combinations}
\question[2] Consider the vectors $[x, y], [a,b]$ and $[c, d]$. 
\begin{parts}
\part Express $[x, y]$ as a linear combination of $[a,b]$ and $[c, d]$.
\begin{solution}

\end{solution}

\part Based on the expression that you have derived above, write down the condition under which $[x, y]$ cannot be expressed as a linear combination of $[a,b]$ and $[c, d]$. (Must: the condition should talk about some relation between the scalars $a,b,c,d,x$ and $y$)
\begin{solution}

\end{solution}

\end{parts}


\uplevel{\textbf{Concept}:  Elementary matrices}

\question[1] Consider the matrix $E_{r+\alpha q}$ that represents the elementary row operation of adding a multiple of $\alpha$ times row $q$ to row $r$.

\uplevel{Under what conditions is $E_{r+\alpha q}$}
\begin{parts}
\part upper triangular? %(hint: a condition involving $r$ and $q$)
\begin{solution}

\end{solution}

\part lower triangular? %(hint: a condition involving $r$ and $q$)
\begin{solution}

\end{solution}
\end{parts}

\question[1] Let $E_1, E_2, E_3, \dots, E_n$ be $n$ elementary matrices. Let $(i_1, j_1), (i_2, j_2), \dots (i_n, j_n)$ be the position of the non-zero off-diagonal element in each of these elementary matrices. Further, $if~k \neq m~then~(i_k, j_k) \neq (i_m, j_m)$ (\textit{i.e.}, no two elementary matrices in the sequence have a non-zero off-diagonal element in the same position). Prove that the product of these $n$ elementary matrices will have all diagonal entries as 1. (Proving this will help you understand why the diagonal elements of $L$ are always equal to 1.)
\begin{solution}

\end{solution}

\uplevel{\textbf{Concept}:  Inverse}
\question[\half] If $A$ is a square invertible matrix then prove that the inverse of $A^\top$ is $A^{-1\top}$
\begin{solution}

\end{solution}

\question[2] Prove that a $n \times n$ matrix A is invertible if and only if Gaussian Elimination of A produces $n$ non-zero pivots. 
\begin{solution}
~\\~\\
Proof (the if part): \\
~\\~\\
Proof (the only if part): \\
\end{solution}

\question[1] If $A$ is a $n\times n$ matrix then what is the cost of:
\begin{parts}
\part Computing $A^{-1}$
\begin{solution}

\end{solution}
\part Computing $A^{-1}b$
\begin{solution}

\end{solution}
\end{parts}

\uplevel{\textbf{Concept}:  LU factorisation}
\question [1 \half] In the lecture, we saw that once we do $LU$ factorisation, we can solve $A\mathbf{x} = \mathbf{b}$ by solving two triangular systems $L\mathbf{c} = \mathbf{b}$ and $U\mathbf{x} = \mathbf{c}$.
\begin{parts}
\part Prove that $L\mathbf{c} = \mathbf{b}$.
\begin{solution}

\end{solution}

\part What is the cost of solving a triangular system (say $L\mathbf{c} = \mathbf{b}$ or $U\mathbf{x}=\mathbf{c}$)?
\begin{solution}

\end{solution}

\part Based on the above results can you comment on the utility of LU factorisation?
\begin{solution}
~\\~\\
\textit{One time cost of LU factorisation:} \\~\\
\textit{Recurring cost of solving $L\mathbf{c} = \mathbf{b}$ and $U\mathbf{x} = \mathbf{c}$}: \\~\\
Hence, $\dots$
\end{solution}

\end{parts}

\question[2] Consider the following system of linear equations. Find the $LU$ factorisation of the matrix A corresponding to this system of linear equations. Show all the steps involved. (this is where you will see what happens when you have to do more than 1 permutations).

\begin{align*}
x + y - 2z  &=  -3\\
w + 2x - y         &= +2\\
w - 4x - 7y  -z   &= -19\\
2w + 4x + y -3z  &=  -2 
\end{align*}

\begin{solution}

\end{solution}

\question[1 \half] For a square matrix A:

\begin{parts}
\part Prove or disprove: $LU$ factorisation is unique.
\begin{solution}

\end{solution}

\part Prove or disprove: $LDU$ factorisation is unique.
\begin{solution}

\end{solution}
\end{parts}

\question[1 \half] Consider the matrix $A$ which factorises as:

$\begin{bmatrix}
1 & 0 & 0\\
2 & 1 & 0\\
0 & 5 & 1
\end{bmatrix}
\begin{bmatrix}
1 & 2 & 0\\
0 & 1 & 5\\
0 & 0 & 1
\end{bmatrix}$

\uplevel{Without computing $A$ or $A^{-1}$ argue that}

\begin{parts}
\part A is invertible (I am looking for an argument which relies on a fact about elementary matrices)
\begin{solution}

\end{solution}
\part A is symmetric (convince me that $A_{ij} = A_{ji}$ without computing $A$)
\begin{solution}

\end{solution}
\part A is tridiagonal (again, without computing $A$ convince me that all elements except along the 3 diagonals will be 0.)
\begin{solution}

\end{solution}

\end{parts}



\uplevel{\textbf{Concept}: Lines and planes}
\question[1 \half] Consider the following system of linear equations 
\begin{align*}
    a_1x_1 + b_1y_1 + c_1z_1 & = 1\\
    a_2x_2 + b_2y_2 + c_2z_2 & = 0\\
    a_3x_3 + b_3y_3 + c_3z_3 & = -1
\end{align*}
Each equation represents a plane, so find out the values for the coefficients such that the following conditions are satisfied:\\
\begin{enumerate}
    \item All planes intersect at a line
    \item All planes intersect at a point
    \item Every pair of planes intersects at a different line.
\end{enumerate}
\begin{solution}

\end{solution}

\question [1 \half]  Starting with a first plane $u+2v-w=6$, find the equation for
\begin{parts}
\part the parallel plane through the origin.
\begin{solution}

\end{solution}
\part a second plane that also contains the points (6,0,0) and (2,2,0). 
\begin{solution}

\end{solution}
\part a third plane that meets the first and second in the point (4, 1, 0).
\begin{solution}

\end{solution}

\end{parts}

\uplevel{\textbf{Concept}: Transpose}
\question[2] Consider the transpose operation.

\begin{parts}
\part Show that it is a linear transformation.
\begin{solution}

\end{solution}

\part Find the matrix corresponding to this linear transformation.
\begin{solution}

\end{solution}

\end{parts}


\end{questions}
\end{document} 