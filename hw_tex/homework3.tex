\documentclass[solution,addpoints,12pt]{exam}
\usepackage{amsmath}
\usepackage{amsthm}
\usepackage{amssymb}
\usepackage{tikz}
\usepackage{animate}
\usepackage{hyperref}

\newtheorem{theorem}{Theorem}
\newtheorem{lemma}[theorem]{Lemma}

\newenvironment{Solution}{\begin{EnvFullwidth}\begin{solution}}{\end{solution}\end{EnvFullwidth}}

\printanswers
%\unframedsolutions
\pagestyle{headandfoot}

%%%%%%%%%%%%%%%%%%%%%%%%%%%%%%%%%%%%%%%%%%%%%%%%%%%%%%
%%%%%%%%%%%%%%%%%%% INSTRUCTIONS %%%%%%%%%%%%%%%%%%%%%
% * Fill in your name and roll number below

% * Answer in place (after each question)

% * Use \begin{solution} and \end{solution} to typeset
%   your answers.
%%%%%%%%%%%%%%%%%%%%%%%%%%%%%%%%%%%%%%%%%%%%%%%%%%%%%%
%%%%%%%%%%%%%%%%%%%%%%%%%%%%%%%%%%%%%%%%%%%%%%%%%%%%%%

% Fill in the details below
\def\studentName{\textbf{Name: TODO}}
\def\studentRoll{\textbf{Roll No: TODO}}

\firstpageheader{CS 6015 (LARP) - Homework 3}{}{\studentName,\studentRoll}
\firstpageheadrule

\newcommand{\brac}[1]{\left[ #1 \right]}
\newcommand{\curly}[1]{\left\{ #1 \right\}}
\newcommand{\paren}[1]{\left( #1 \right)}
\newcommand{\card}[1]{\left\lvert #1 \right\rvert}

\begin{document}

\noindent \textbf{Honor code}: I pledge on my honor that: I have completed all steps in the below homework on my own, I have not used any unauthorized materials while completing this homework, and I have not given anyone else access to my homework.
\\~\\~\\
\begin{flushright}
\textbf{Name and Signature}

\end{flushright}


\begin{questions}

\question[1] Have you read and understood the honor code?
\begin{solution}

\end{solution}

\uplevel{\textbf{Concept}:  System of linear equations}


\question[2] This question has two parts as mentioned below:

\begin{parts}
\part Find a 2 x 3 system Ax = b whose complete solution is
\[ x =
\begin{bmatrix}
1 \\
2 \\
0
\end{bmatrix}
%\]
+
w
\begin{bmatrix}
1\\
3\\
1
\end{bmatrix}
\]
\begin{solution}

\end{solution}
\part Now find a 3 x 3 system which has these solutions exactly when $b_1 + b_2 = b_3$. (Note: $b=[b_1\ b_2\  b_3]^T$.)
\begin{solution}

\end{solution}
\end{parts}

\question[2] Consider the matrices $A$ and $B$ below\\
\\
(i) A = $
\begin{bmatrix}
1 & 2 & 0 & 1 \\
0 & 1 & 1 & 0 \\
1 & 2 & 0 & 1 
\end{bmatrix}
$
(ii) B = $
\begin{bmatrix}
1 & 2 & 3 \\
4 & 5 & 6 \\
7 & 8 & 9
\end{bmatrix}
$
\\
\begin{parts}
\part Write down the row reduced echelon form of matrices $A$ and $B$ (also mention the steps involved).
\begin{solution}

\end{solution}

\part Find all solutions to $A\mathbf{x} = 0$ and $B\mathbf{x} = 0$. 
\begin{solution}

\end{solution}

\part Write down the basis for the four fundamental subspaces of $A$.
\begin{solution}

\end{solution}
\part Write down the basis for the four fundamental subspaces of $B$.
\begin{solution}

\end{solution}
\end{parts}
\uplevel{\textbf{Concept}:  Rank}
\question[1 \half] Consider the matrices $A$ and $B$ as given below:

~\\
 $A =
%\[
\begin{bmatrix}
6 & 4 & 2  \\
-3 & -2 & -1 \\
9 & 6 & x
\end{bmatrix}
%\]
$
and B =
$
\begin{bmatrix}
3 & 1 & 3 \\
y & 2 & y
\end{bmatrix}
$
~\\

Give the values for entries $x$ and $y$ such that the ranks of the matrices $A$ and $B$ are 
\begin{parts}
\part 1
\begin{solution}

\end{solution}

\part 2
\begin{solution}

\end{solution}

\part 3
\begin{solution}

\end{solution}

\end{parts}


\uplevel{\textbf{Concept}:  Nullspace and column space}
\question[\half] State True or False and explain you answer: The nullspace of $R$ is the same as the nullspace of $U$ (where $R$ is the row reduced echelon form of $A$ and $U$ is the matrix in $LU$ decomposition of $A$). 
\begin{solution}
True/False because ...
\end{solution}

\question [1] Suppose column 1 + column 2 + column5 = $\mathbf{0}$ in a 4 $\times$ 5 matrix $A$. 
\begin{parts}
\part What is a special solution for $A\mathbf{x} = \mathbf{0}$
\begin{solution}

\end{solution}
\part Describe the null space of $A$.
\begin{solution}

\end{solution}
\end{parts}

\question[2] Consider the matrix $A=$
$\begin{bmatrix}
1&1\\
1&2\\
1&3\\
\end{bmatrix}$. The column space of this matrix is a 2 dimensional plane. What is the equation of this plane? (You need to write down the steps you took to arrive at the equation)
\begin{solution}

\end{solution}
\question[1] True or false? (If true give logical, valid reasoning or give a counterexample if false)\\
a. If the row space equals the column space then $A^T = A$
\begin{solution}
\end{solution}
b. If $A^T$ = $-A$ then the row space of A equals the column space.
\begin{solution}
\end{solution}

\question[1] Which of the four fundamental subspaces are the same for the following pairs of matrices of different sizes? (Assume $A$ is a $m \times n$ matrix)\\
\begin{parts}
\part $ \begin{bmatrix}
A
\end{bmatrix}
\text{ and }
\begin{bmatrix}
A\\A
\end{bmatrix}$
\begin{solution}


\end{solution}
\part $ \begin{bmatrix}
A\\A
\end{bmatrix}
\text{ and }
\begin{bmatrix}
A & A\\A & A
\end{bmatrix}$
\begin{solution}

\end{solution}

\end{parts}

\question[2] For each of the questions below, construct a matrix $A$ which satisfies the given condition or argue why the given condition cannot be satisfied? 

\begin{parts}
\part A matrix whose row space is equal to its column space 
\begin{solution}

\end{solution}

\part A matrix whose null space is equal to its column space 
\begin{solution}

\end{solution}

\part A matrix for which all the four fundamental subspaces are equal
\begin{solution}

\end{solution}
\end{parts}

\question[1] True or false? If $A$ is a $m\times m$ square matrix then $\mathcal{N}(A) = \mathcal{N}(A^2)$ (If true give logical, valid reasoning or give a counterexample if false)
\begin{solution}

\end{solution}

\question[2] Consider matrices $A$ and $B$ and their product $AB$. For each of the questions below fill in the blanks with one of the following options: $<,>,=, \leq,\geq,~can't~say$. Explain your answer.
\begin{parts}

\part $dim(\mathcal{C}(AB)) \_\_\_\_\_\_\_\_\_\_\_\_\_\_\_\_  dim(\mathcal{C}(A))$
\begin{solution}

\end{solution}

\part $dim(\mathcal{C}(AB)) \_\_\_\_\_\_\_\_\_\_\_\_\_\_\_\_  dim(\mathcal{C}(B))$
\begin{solution}

\end{solution}

\part $dim(\mathcal{C}((AB)^\top)) \_\_\_\_\_\_\_\_\_\_\_\_\_\_\_\_  dim(\mathcal{C}(A^\top))$
\begin{solution}

\end{solution}

\part $dim(\mathcal{C}((AB)^\top)) \_\_\_\_\_\_\_\_\_\_\_\_\_\_\_\_  dim(\mathcal{C}(B^\top))$
\begin{solution}

\end{solution}

\end{parts}

\uplevel{\textbf{Concept}:  Free variables}
\question[2 \half] True or False (with reason if true or example to show it is false).
\begin{parts}
\part A square matrix has no free variables
\begin{solution}
True/False because ...
\end{solution}
\part An invertible matrix has no free variables
\begin{solution}
True/False because ...
\end{solution}
\part An $m \times n$ matrix has no more than $n$ pivot variables. 
\begin{solution}
True/False because ...
\end{solution}
\part An $m \times n$ matrix has no more than $m$ pivot variables.
\begin{solution}
True/False because ...
\end{solution}
\part Matrices $A$ and $A^T$ have the same number of pivots.
\begin{solution}
True/False because ...
\end{solution}
\end{parts}

\uplevel{\textbf{Concept}:  Reduced Echelon Form}


\question[\half] Suppose R is $m \times n$ matrix of rank $r$, with pivot columns first:

\begin{equation*}
R = 
\begin{bmatrix}
I & F\\
0 & 0
\end{bmatrix}
\end{equation*}

\begin{parts}

\part Find a right-inverse $B$ with $RB = I$ if $r = m$. 
\begin{solution}

\end{solution}

\end{parts}

\end{questions}
\end{document} 